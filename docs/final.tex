\documentclass{article}
\usepackage{tikz}
\usetikzlibrary{calc}
\newcommand{\tikzmark}[1]{\tikz[overlay,remember picture] \node (#1) {};}
\newcommand{\underarrow}[2] {
  \begin{tikzpicture}[overlay,remember picture,out=340,in=210,distance=0.3cm]
    \draw [->,shorten >=3pt,shorten <=-3pt] ({#1}.south) to ({#2}.west);
  \end{tikzpicture}
}  
\newcommand{\Ex}{\mathbb{E}}
\begin{document}

\subsection{Non-Simply-Connected Graphs}

What if there is a causal effect $A\rightarrow C$?  It must be weak
enough not to be detected, but that doesn't say much.  Let us consider
it by cases.

\subsubsection{False Colliders}

Can a true graph of
$A\tikzmark{a}\rightarrow B \rightarrow{C}\tikzmark{c}$
\underarrow{a}{c}
produce a $p(A,C)$ more similar to a graph
$A \rightarrow B \leftarrow C$?  This would require the direct and
indirect influence of $A$ on $C$ to cancel out rather precisely.  If
the indirect influence is stronger, we will pick the correct model,
albeit underconfidently.  If the direct influence is too strong, we
will see a correlation between $A$ and $C$.  And if the direct
influence is in the same direction as the indirect, the correct model
will remain the better fit (though both models will be worse).


\begin{eqnarray*}
\prod_{a,c} p(a,c|m_{co})^{\Ex n_{a,c}} & > & \prod_{a,c} p(a,c|m_{ch})^{\Ex n_{a,c}} \\
\sum_{a,c} \log(p(a,c|m_{co}))\Ex n_{a,c} & > & \sum_{a,c} \log(p(a,c|m_{ch}))\Ex n_{a,c} \\


\end{document}
